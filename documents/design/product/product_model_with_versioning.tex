\documentclass[11pt]{article}
\usepackage[utf8]{inputenc}
\usepackage[english]{babel}
\usepackage{hyperref}
\usepackage{listings}
%for dotex
\usepackage[pdftex]{graphicx}
\usepackage[pdf]{graphviz}
\usepackage[boxed]{algorithm2e} %end for dote
\usepackage{color}

% "define" Scala
\lstdefinelanguage{scala}{
  morekeywords={abstract,case,catch,class,def,%
    do,else,extends,false,final,finally,%
    for,if,implicit,import,match,mixin,%
    new,null,object,override,package,%
    private,protected,requires,return,sealed,%
    super,this,throw,trait,true,try,%
    type,val,var,while,with,yield},
  otherkeywords={=>,<-,<\%,<:,>:,\#,@},
  sensitive=true,
  morecomment=[l]{//},
  morecomment=[n]{/*}{*/},
  morestring=[b]",
  morestring=[b]',
  morestring=[b]"""
}

\lstset{ %
    language=scala,
    identifierstyle=\textbf
}

\title{Product Model Versioning}
\author{Maxim Noah Khailo}
\begin{document}
\maketitle
\section{Purpose}

The product model described in "The Enlightened, Post-Modern Product Model" 
article went a long way describing a way to manage change based on context.
This article takes the design and pushes it further to include versioning.

Because not only does the product change in context, but also changes in time.

\section{Forces}

We have to balance several problems.

\begin{itemize}
    \item Provide Change in context and time.
    \item Handle various UI paradigms for showing differences and undo.
    \item Optimize data for storefront while maintaining control in PIM.
\end{itemize}

\section{The Shadow Change Log}

The Product Form is composed of named attributes. Each attribute has a type, and
a set of versions. A shadow points to specific versions of attributes on a form. 

We can represent versioning a product as a log of product shadows. Here we introduce
the concept of a product commit, which points to a shadow. Since a shadow is a
"snapshot" of a product form, we can represent all history to changes of a product
within a specific context over time.

\digraph{Parts} {
    node [shape=record];
    Context [label="Context"];
    {
        rank=same;
        Head1[label="..."];
        Head2 [label="..."];
        Head3 [label="Head"];
    }
    {
        rank=same;
        Commit1 [label="Commit 1"]
        Commit2 [label="Commit 2"]
        Commit3 [label="Commit 3"]
    }
    {
        rank=same;
        Shadow1 [label="Shadow 1"];
        Shadow2 [label="Shadow 2"];
        Shadow3 [label="Shadow 3"];
    }
    Form [label="Product Form"];

    Context -> Head1 [label="Has"];
    Context -> Head2 [label="Has"];
    Context -> Head3 [label="Has"];

    Head3 -> Commit3 [label="Points to"];
    Commit1 -> Commit2[dir="back"];
    Commit2 -> Commit3[dir="back"];
    Commit1 -> Shadow1;
    Commit2 -> Shadow2;
    Commit3 -> Shadow3;
    Shadow1 -> Form 
    Shadow2 -> Form 
    Shadow3 -> Form [label="Snapshot of"]
}

\section{Changes to Product Form and Shadow}

The changes to the product model are that the named attributes have a set 
of versions of the attribute. Before we used named keys but why not just
set the key name to a hash of the content?

Here is the old model...

\begin{lstlisting}
    //Old Model Product 
    {
        id: 3,
        name: { 
            type: "string", 
            x: "Red Shoe", 
            y: "Big Red Shoe", 
        },
    }
    //Old Shadow
    {
        productId: 3,
        name : "y"
    }
\end{lstlisting}

Instead of "x" and "y", we compute the SHA-1 hash of the content and take the
first x values of the hash. Lets say we take the first 7.

\begin{lstlisting}
    //New Model Product 
    {
        id: 3,
        "d5ae001": "Red Shoe", 
        "2981c9f": "Big Red Shoe", 
    }
    //New Product Shadow
    {
        productId: 3,
        name : {type:"string", ref: "2981c9f"}
    }
\end{lstlisting}

Notice here, all a shadow is is a pointer to versions of attributes in the
product form. Each attribute in the shadow has metadata about the type and a 
reference to the attribute data from the form. The idea here is that the data
should out weigh the metadata by a lot. 

If we use the git revision control system as an analogy, the product 
shadow is similar a git tree object and the product form is a collection of 
attributes, where each attribute has a set of versions of that attribute.


\section{The Product Commit Object}

Since each shadow is a "snapshot" of a form, an ordered list of shadows that 
point to the same form could represent a history of changes of that product.

\begin{lstlisting}
    //Commit Object
    {
        id: 3,
        shadowId: 5,
        productId: 4,
        previousCommitId: 2,
        reasonId: 10
    }
\end{lstlisting}

\subsection{Keeping Track of the Latest Shadow in a Context}

To keep track of the set of latest shadows, we need another object that simply
points to the latest commit. We will call the Head Object

\begin{lstlisting}
    //Head Object
    {
        id: 3,
        contextId: 1,
        shadowId: 5,
        productId: 4,
        commitId: 3
    }
\end{lstlisting}

We can also store the \emph{shadowId} and \emph{productId} here for convenience.

\section{Discussion}

There are some pros and cons to this design. 

\subsection{Pros}

\begin{itemize}
    \item The Form has all the versions of each attribute together. This should
        compress well.
    \item The Shadow is decoupled from the context. We can share a shadow into
        a context by creating a Head Object.
    \item The commit log is a tree. We can write merge algorithms later if we wanted
        to.
    \item We can use contexts the same way people uses branches in git. An admin
        can have their own context where they stage commits. Once they are approved
        The head object in the stores context can point to the correct commit.
    \item We can have contexts based on events. For example a context to stage
        commits for Easter and migrate the store to easter mode, then migrate
        back to normal mode trivially.
    \item Using contexts and head objects, we can trivially snapshot the store
        or provide custom catalogs. 
\end{itemize}

\subsection{Cons}
\begin{itemize}
    \item PIM UI, which is shadow aware now, will need to be commit aware.
    \item Is a full git type system in the DB a good idea? Could a form get too big?
    \item Product form makes no sense without a shadow, since all the metadata is
        on the shadow. The form becomes a collection or attribute data for a product.
\end{itemize}
\end{document}


