\documentclass[11pt]{article}
\usepackage[utf8]{inputenc}
\usepackage[english]{babel}
\usepackage{hyperref}
\usepackage{listings}
%for dotex
\usepackage[pdftex]{graphicx}
\usepackage[pdf]{graphviz}
\usepackage[boxed]{algorithm2e} %end for dote
\usepackage{color}
\usepackage{morewrites}

\title{Multi Tenancy to Completion}
\begin{document}
\maketitle
\section{The Problem}

FoxCommerce platform has been moving towards multi-tenancy and is almost there. 
There are a couple of remaining issues that need to be addressed for the system 
to be completely multi tenant. 

\section{What We Have}

\subsection{Organizations}

We have a model of organizations which have roles and permissions and users.
Organizations are associated with a specific scope. Scopes provide a way to
organize data hierarchically and limit access.

\subsubsection{Capabilities and Constraints}

\paragraph{Organizations can...}

\begin{enumerate}
    \item Control roles and permissions.
    \item Control how a scope is accessed.
    \item Have sub organizations that belong to subscopes.
    \item Control subscopes.
\end{enumerate}

\paragraph{Organizations cannot...}

\begin{enumerate}
    \item Cross sibling scopes.
    \item Have unscoped data.
    \item Users cannot log into multiple scopes at same time.
\end{enumerate}

\subsection{Scopes}

Almost all tables in the system have a scope column. Scopes are hierarchical 
organization of data like a filesystem tree. Users with access to a scope do not
have access to the parent scope.

\digraph[scale=0.80]{Scopes}{
    splines="ortho";
    rankdir=LR;
    node [shape=box,style=filled,fillcolor="lightblue"];

    subgraph zero{
        tenant [shape=record, label="Tenant (1)"]
    };
    subgraph first{
        merchant1 [shape=record,label="{Merchant A(1.2)}"];
        merchant2 [shape=record,label="{Merchant B(1.3)}"];
    };

    tenant -> merchant1
    tenant -> merchant2
}

Each merchant may have one or more storefronts. The question of whether the
data of those storefronts is scoped depends on the use cases we want to enable.
Is a different organization managing the other store fronts? Then we probably want this

\digraph[scale=0.80]{Storefronts}{
    splines="ortho";
    rankdir=LR;
    node [shape=box,style=filled,fillcolor="lightblue"];

    subgraph zero{
        tenant [shape=record, label="Tenant (1)"]
    };
    subgraph first{
        merchant1 [shape=record,label="{Merchant A(1.2)}"];
        merchant2 [shape=record,label="{Merchant B(1.3)}"];
    };
    subgraph third {
        storefront1 [shape=record,label="{Storefront A1(1.2.4?)}"];
        storefront2 [shape=record,label="{Storefront A2(1.2.5?)}"];
        storefront3 [shape=record,label="{Storefront B(1.3.6?)}"];
    };

    tenant -> merchant1
    tenant -> merchant2
    merchant1 -> storefront1
    merchant1 -> storefront2
    merchant2 -> storefront3
}

Is the same organization managing various storefronts? Then we want this.

\digraph[scale=0.80]{Storefronts}{
    splines="ortho";
    rankdir=LR;
    node [shape=box,style=filled,fillcolor="lightblue"];

    subgraph zero{
        tenant [shape=record, label="Tenant (1)"]
    };
    subgraph first{
        merchant1 [shape=record,label="{Merchant A(1.2)}"];
        merchant2 [shape=record,label="{Merchant B(1.3)}"];
    };
    subgraph third {
        storefront1 [shape=record,label="{Storefront A1(1.2)}"];
        storefront2 [shape=record,label="{Storefront A2(1.2)}"];
        storefront3 [shape=record,label="{Storefront B(1.3)}"];
    };

    tenant -> merchant1
    tenant -> merchant2
    merchant1 -> storefront1
    merchant1 -> storefront2
    merchant2 -> storefront3
}

Notice that the scope of the storefronts is the same. Regardless if we have one
organization or another we need a different organizing structure for storefront
data that is separate from scopes. We want a model of channels.

\subsubsection{Capabilities and Constraints}

\paragraph{Scopes can...}

\begin{enumerate}
    \item Group data like a directory in a filesystem.
    \item Control access to data via roles/permissions that are in that scope.
\end{enumerate}

\paragraph{Scopes cannot...}

\begin{enumerate}
    \item Share data with sibling scopes.
    \item Provide semantic relationships between data in a scope.
    \item Provide semantic relationships between data in different scopes.
\end{enumerate}

\subsection{Views (formally Context)}

All merchandising information can have several views. Views provide a way to change the 
information of a product, sku, discount, or other merchandising data for a specific
purpose. For example, each storefront could possibly have a different view of a product.
A review/approval flow may have it's own view.

\digraph[scale=0.80]{Views}{
    splines="ortho";
    rankdir=LR;
    node [shape=record,style=filled,fillcolor="lightblue"];

    subgraph zero{
        product [label="Product"]
    };
    subgraph first{
        view1 [label="{View for Storefront A}"];
        view2 [label="{View for Storefront B}"];
        view3 [label="{View for Review/Approval}"];
    };

    product -> view1
    product -> view2
    product -> view3
}

\subsubsection{Capabilities and Constraints}

\paragraph{Views can...}

\begin{itemize}
    \item Control which versions of data in the object store are displayed.
    \item Act as a git branch on the merchandising data.
    \item Commits provide branching history between views.
\end{itemize}


\paragraph{Views cannot...}

\begin{itemize}
    \item Control access to data.
    \item Shared between sibling scopes.
    \item Cannot control versions of parts of objects.
    \item Cannot describe semantic relationships between views.
\end{itemize}

\section{What We Need}

\subsection{Catalogs}
Catalogs are collections of products you want to sell.

\digraph[scale=0.80]{Catalogs}{
    splines="ortho";
    rankdir=LR;
    node [shape=box,style=filled,fillcolor="tan"];

    subgraph zero{
        catalog [label="Catalog"]
    };
    subgraph first{
        scope [label="Scope"];
        products [label="Products"];
        discounts [label="Discounts"];
        name [label="Name"];
        country [label="Country"];
        language [label="Language"];
        live [label="Live View"];
        stage [label="Stage View"];
    };

    catalog -> scope [label="belongs to"]
    catalog -> name [label="has a"]
    catalog -> products [label="has"]
    catalog -> discounts [label="has"]
    catalog -> payment [label="has"]
    catalog -> country [label="for a"]
    catalog -> language [label="has default"]
    catalog -> live [label="points to"]
    catalog -> stage [label="points to"]
}

\subsection{Channels}

Channels should be comprised of three key components

\digraph[scale=0.80]{Channels}{
    splines="ortho";
    rankdir=LR;
    node [shape=box,style=filled,fillcolor="tan"];

    subgraph zero{
        channel [label="Channel"]
    };
    subgraph first{
        scope [label="Scope"];
        catalog [label="Catalog"];
        payment [label="Payment Methods"];
        stock [label="Stock"];
        aux [label="Auxiliary Data..."];
    };

    channel -> scope [label="belongs to"]
    channel -> catalog [label="has a"]
    channel -> payment [label="uses"]
    channel -> stock [label="has"]
    channel -> aux [label="has"]
}

\subsection{Storefronts}

A storefront is a website that can sell products from a catalog. A storefront
uses a channel in addition to data from the CMS.

\digraph[scale=0.80]{StorefrontModel}{
    splines="ortho";
    rankdir=LR;
    node [shape=box,style=filled,fillcolor="tan"];

    subgraph zero{
        store [label="Storefront(Channel)"]
    };
    subgraph first{
        scope [label="Scope"];
        channel [label="Channel"];
        live [label="Live CMS View"];
        stage [label="Stage CMS View"];
        host [label="Host"];
    };

    store -> scope [label="belongs to"]
    store -> channel [label="uses"]
    store -> live [label="shows"]
    store -> stage [label="points to"]
    store -> host [label="serves from"]
}

\newpage
\subsection{Back to Organizations}

Once we have these new models we can assign them to organizations

\digraph[scale=0.80]{OrgModel}{
    splines="ortho";
    rankdir=TD;
    node [shape=box,style=filled,fillcolor="tan"];

    subgraph zero{
        rank=source;
        org [label="Organization"]
    };

    subgraph first{
        scope [label="Scope 1.2"]
    };

    subgraph second{
        rank=same;
        catalog [label="Master Catalog"];
        view [label="Master View"];
    };
    subgraph third{
        view1 [label="View A"];
        view2 [label="View B"];
    };

    subgraph fourth{
        catalog1 [label="Catalog A"];
        catalog2 [label="Catalog B"];
    };

    subgraph fifth{
        channel1 [label="Channel A"];
        channel2 [label="Channel B"];
    };

    subgraph sixth {
        rank=sink;
        store1 [label="Storefront A"];
        store2 [label="Storefront B"];
    };

    org -> scope [label="belongs to"]
    scope -> catalog;
    scope -> view;
    catalog -> catalog1
    catalog -> catalog2
    view -> view1 [label="branch"];
    view -> view2 [label="branch"];
    view1 -> catalog1
    view2 -> catalog2
    catalog1 -> channel1 [label="uses"];
    catalog2 -> channel2 [label="uses"];
    store1 -> channel1 [label="used by"];
    store2 -> channel2 [label="used by"];
}

\subsection {Public Access}

Our current public endpoints for searching the catalog, registering users, and
viewing/purchasing products are not multi-tenant aware at the moment. We need to 
modify them to understand which channel they are serving.


\digraph[scale=0.80]{Proxy}{
    splines="ortho";
    rankdir=TD;
    node [shape=record,style=filled,fillcolor="lightblue"];

    client [label="Client"];
    riverrock [label="River Rock Proxy"];
    db [shape=note,label="DB"];
    channel [label="Channel for host.com"];
    catalog [label="Catalog"];
    view [label="View"];

    client -> riverrock [label="request host.com"]
    riverrock -> db [label="find channel"]
    db -> channel [label="found"];
    channel -> view;
    channel -> catalog;
    view -> riverrock [label="serve"];
    catalog -> riverrock [label="serve"];

}

\section{Example Story}

We acquired a customer which sells shoes called ShoeMe that wants to operate two
global sites and sell products on amazon. They want to sell on two domains,
shoeme.co.uk and shoeme.co.jp and amazon.com.

\digraph[scale=0.80]{ShoeMe}{
    splines="ortho";
    rankdir=LR;
    node [shape=record,style=filled,fillcolor="pink"];

    subgraph zero{
        shoeme [label="ShoeMe"]
    };
    subgraph first{
        amazon [label="{amazon.com}"];
        uk [label="{shoeme.co.jp}"];
        jp [label="{shoeme.co.uk}"];
    };

    shoeme -> amazon;
    shoeme -> uk;
    shoeme -> jp;
}

We create a new \emph{Organization} called ShoeMe and \emph{Scope} with id ``1.2'' .
Within that \emph{Organization} we create a \emph{role} called ``admin'' and a new
\emph{user} assigned to that role. 

\digraph[scale=0.80]{ShoeMeOrg}{
    splines="ortho";
    rankdir=TD;
    node [shape=record,style=filled,fillcolor="pink"];

    org [label="ShoeMe Org"]
    scope [label="Scope 1.2"];
    admin [label="{Admin Role|Admin Users}"];

    scope -> org;
    org -> admin;
}

The ShoeMe representative follows the password creation flow for their new account and
gets access to ashes.

We create for the customer four catalogs called ``master'', ``uk'', ``jp'', and ``amazon''.
Behind the scenes with creates four views ``master'', ``uk'', ``jp'', and ``amazon''.
They then import there products into the master catalog. They then use ashes to selectively
assign products into the other three catalogs. This forks the products into the three other
views. 

\digraph[scale=0.80]{ShoeMeCatalogs}{
    splines="ortho";
    rankdir=TD;
    node [shape=record,style=filled,fillcolor="pink"];

    scope [label="{Scope 1.2}"];

    master [fillcolor="green",label="{Master|{Catalog|View}}"];

    subgraph catalogs { 
        rank=same;
        uk [fillcolor="green",label="{UK|{Catalog|View}}"];
        jp [fillcolor="green",label="{JP|{Catalog|View}}"];
        amazon [fillcolor="green",label="{Amazon|{Catalog|View}}"];
    };

    scope -> master;
    master -> uk;
    master -> jp;
    master -> amazon;
}

We create three channels for the customer called ``uk''. ``jp''. and ``amazon'' and assign
each appropriate catalog to the channels.

\digraph[scale=0.80]{ShoeMeChannels}{
    splines="ortho";
    rankdir=TD;
    node [shape=record,style=filled,fillcolor="pink"];

    scope [label="{Scope 1.2}"];
    master [fillcolor="green",label="{Master|{Catalog|View}}"];

    subgraph catalogs { 
        rank=same;
        uk [fillcolor="green",label="{UK|{Catalog|View}}"];
        jp [fillcolor="green",label="{JP|{Catalog|View}}"];
        amazon [fillcolor="green",label="{Amazon|{Catalog|View}}"];
    };

    subgraph channels { 
        rank=same;
        ukc [fillcolor="lightblue",label="{UK|{Channel|Stock}}"];
        jpc [fillcolor="lightblue",label="{JP|{Channel|Stock}}"];
        amazonc [fillcolor="lightblue",label="{Amazon|{Channel|Stock}}"];
    };

    scope -> master;
    master -> uk;
    master -> jp;
    master -> amazon;
    uk -> ukc;
    jp -> jpc;
    amazon -> amazonc;
}


We then create two storefronts, ``shoeme.co.uk'' and ``shoeme.co.jp''.
Behind the scenes these two storefronts will create two views which will be used
by the CMS to show various content. Just like with catalogs they can first build one storefront
and then copy over the data into the other.

\digraph[scale=0.80]{ShoeMeStores}{
    splines="ortho";
    rankdir=TD;
    node [shape=record,style=filled,fillcolor="pink"];

    scope [label="{Scope 1.2}"];
    master [fillcolor="green",label="{Master|{Catalog|View}}"];

    subgraph catalogs { 
        rank=same;
        uk [fillcolor="green",label="{UK|{Catalog|View}}"];
        jp [fillcolor="green",label="{JP|{Catalog|View}}"];
        amazon [fillcolor="green",label="{Amazon|{Catalog|View}}"];
    };

    subgraph channels { 
        rank=same;
        ukc [fillcolor="lightblue",label="{UK|{Channel|Stock}}"];
        jpc [fillcolor="lightblue",label="{JP|{Channel|Stock}}"];
        amazonc [fillcolor="lightblue",label="{Amazon|{Channel|Stock}}"];
    };

    subgraph storefronts { 
        rank=same;
        uks [fillcolor="yellow",label="{shoeme.co.uk|{Storefront|CMS View}}"];
        jps [fillcolor="yellow",label="{shoeme.co.jp|{Storefront|CMS View}}"];
    };

    scope -> master;
    master -> uk;
    master -> jp;
    master -> amazon;
    uk -> ukc;
    jp -> jpc;
    amazon -> amazonc;
    ukc -> uks;
    jpc -> jps;
}

Our API will understand how to serve traffic depending on the host of each storefront 
by selecting the appropriate CMS views and catalog assigned to the storefront.

Behind the scenes will can also provide sister views to each view for staging changing
which await approval.

Each catalog may also be assigned the same or different stock item location and
pricing information. Each catalog can therefore share or maintain separate stock
for the products in the catalog.


\end{document}
