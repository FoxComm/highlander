\documentclass[11pt]{article}
\usepackage[utf8]{inputenc}
\usepackage[english]{babel}
\usepackage{hyperref}
\usepackage{listings}
%for dotex
\usepackage[pdftex]{graphicx}
\usepackage[pdf]{graphviz}
\usepackage[boxed]{algorithm2e} %end for dote
\usepackage{color}

\title{Multi Tenancy to Completion}
\begin{document}
\maketitle
\section{The Problem}

FoxCommerce platform has been moving towards multi-tenancy and is almost there. 
There are a couple of remaining issues that need to be addressed for the system 
to be completely multi tenant. 

\section{What We Have}

\subsection{Organizations}

We have a model of organizations which have roles and permissions and users.
Organizations are associated with a specific scope. Scopes provide a way to
organize data hierarchically and limit access.

\subsubsection{Capabilities and Constraints}

\paragraph{Organizations can...}

\begin{enumerate}
    \item Control roles and permissions.
    \item Control how a scope is accessed.
    \item Have sub organizations that belong to subscopes.
    \item Control subscopes.
\end{enumerate}

\paragraph{Organizations cannot...}

\begin{enumerate}
    \item Cross sibling scopes.
    \item Have unscoped data.
    \item Users cannot log into multiple scopes at same time.
\end{enumerate}

\subsection{Scopes}

Almost all tables in the system have a scope column. Scopes are hierarchical 
organization of data like a filesystem tree. Users with access to a scope do not
have access to the parent scope.

\digraph[scale=0.80]{Scopes}{
    splines="ortho";
    rankdir=LR;
    node [shape=box,style=filled,fillcolor="lightblue"];

    subgraph zero{
        tenant [shape=record, label="Tenant (1)"]
    };
    subgraph first{
        merchant1 [shape=record,label="{Merchant A(1.2)}"];
        merchant2 [shape=record,label="{Merchant B(1.3)}"];
    };

    tenant -> merchant1
    tenant -> merchant2
}

Each merchant may have one or more storefronts. The question of whether the
data of those storefronts is scoped depends on the use cases we want to enable.
Is a different organization managing the other store fronts? Then we probably want this

\digraph[scale=0.80]{Storefronts}{
    splines="ortho";
    rankdir=LR;
    node [shape=box,style=filled,fillcolor="lightblue"];

    subgraph zero{
        tenant [shape=record, label="Tenant (1)"]
    };
    subgraph first{
        merchant1 [shape=record,label="{Merchant A(1.2)}"];
        merchant2 [shape=record,label="{Merchant B(1.3)}"];
    };
    subgraph third {
        storefront1 [shape=record,label="{Storefront A1(1.2.4?)}"];
        storefront2 [shape=record,label="{Storefront A2(1.2.5?)}"];
        storefront3 [shape=record,label="{Storefront B(1.3.6?)}"];
    };

    tenant -> merchant1
    tenant -> merchant2
    merchant1 -> storefront1
    merchant1 -> storefront2
    merchant2 -> storefront3
}

Is the same organization managing various storefronts? Then we want this.

\digraph[scale=0.80]{Storefronts}{
    splines="ortho";
    rankdir=LR;
    node [shape=box,style=filled,fillcolor="lightblue"];

    subgraph zero{
        tenant [shape=record, label="Tenant (1)"]
    };
    subgraph first{
        merchant1 [shape=record,label="{Merchant A(1.2)}"];
        merchant2 [shape=record,label="{Merchant B(1.3)}"];
    };
    subgraph third {
        storefront1 [shape=record,label="{Storefront A1(1.2)}"];
        storefront2 [shape=record,label="{Storefront A2(1.2)}"];
        storefront3 [shape=record,label="{Storefront B(1.3)}"];
    };

    tenant -> merchant1
    tenant -> merchant2
    merchant1 -> storefront1
    merchant1 -> storefront2
    merchant2 -> storefront3
}

Notice that the scope of the storefronts is the same. Regardless if we have one
organization or another we need a different organizing structure for storefront
data that is separate from scopes. We want a model of channels.

\subsubsection{Capabilities and Constraints}

\paragraph{Scopes can...}

\begin{enumerate}
    \item Group data like a directory in a filesystem.
    \item Control access to data via roles/permissions that are in that scope.
\end{enumerate}

\paragraph{Scopes cannot...}

\begin{enumerate}
    \item Share data with sibling scopes.
    \item Provide semantic relationships between data in a scope.
    \item Provide semantic relationships between data in different scopes.
\end{enumerate}

\subsection{Views (formally Context)}

All merchandising information can have several views. Views provide a way to change the 
information of a product, sku, discount, or other merchandising data for a specific
purpose. For example, each storefront could possibly have a different view of a product.
A review/approval flow may have it's own view.

\digraph[scale=0.80]{Views}{
    splines="ortho";
    rankdir=LR;
    node [shape=record,style=filled,fillcolor="lightblue"];

    subgraph zero{
        product [label="Product"]
    };
    subgraph first{
        view1 [label="{View for Storefront A}"];
        view2 [label="{View for Storefront B}"];
        view3 [label="{View for Review/Approval}"];
    };

    product -> view1
    product -> view2
    product -> view3
}

\subsubsection{Capabilities and Constraints}

\paragraph{Views can...}

\begin{itemize}
    \item Control which versions of data in the object store are displayed.
    \item Act as a git branch on the merchandising data.
    \item Commits provide branching history between views.
\end{itemize}


\paragraph{Views cannot...}

\begin{itemize}
    \item Control access to data.
    \item Shared between sibling scopes.
    \item Cannot control versions of parts of objects.
    \item Cannot describe semantic relationships between views.
\end{itemize}

\section{What We Need}

\subsection{Catalogs}
Catalogs are collections of products you want to sell.

\digraph[scale=0.80]{Catalogs}{
    splines="ortho";
    rankdir=LR;
    node [shape=box,style=filled,fillcolor="tan"];

    subgraph zero{
        catalog [label="Catalog"]
    };
    subgraph first{
        scope [label="Scope"];
        products [label="Products"];
        stock [label="Stock"];
        discounts [label="Discounts"];
        name [label="Name"];
        country [label="Country"];
        language [label="Language"];
        live [label="Live View"];
        stage [label="Stage View"];
    };

    catalog -> scope [label="belongs to"]
    catalog -> name [label="has a"]
    catalog -> products [label="has"]
    catalog -> discounts [label="has"]
    catalog -> stock [label="has"]
    catalog -> payment [label="has"]
    catalog -> country [label="for a"]
    catalog -> language [label="has default"]
    catalog -> live [label="points to"]
    catalog -> stage [label="points to"]
}

\subsection{Channels}

Channels should be comprised of three key components

\digraph[scale=0.80]{Channels}{
    splines="ortho";
    rankdir=LR;
    node [shape=box,style=filled,fillcolor="tan"];

    subgraph zero{
        channel [label="Channel"]
    };
    subgraph first{
        scope [label="Scope"];
        catalog [label="Catalog"];
        payment [label="Payment Methods"];
        payment [label="Payment Methods"];
        aux [label="Auxiliary Data..."];
    };

    channel -> scope [label="belongs to"]
    channel -> catalog [label="has a"]
    channel -> payment [label="uses"]
    channel -> aux [label="has"]
}

\subsection{Storefronts}

A storefront is a website that can sell products from a catalog. A storefront
uses a channel in addition to data from the CMS.

\digraph[scale=0.80]{StorefrontModel}{
    splines="ortho";
    rankdir=LR;
    node [shape=box,style=filled,fillcolor="tan"];

    subgraph zero{
        store [label="Storefront(Channel)"]
    };
    subgraph first{
        scope [label="Scope"];
        channel [label="Channel"];
        live [label="Live CMS View"];
        stage [label="Stage CMS View"];
        host [label="Host"];
    };

    store -> scope [label="belongs to"]
    store -> channel [label="uses"]
    store -> live [label="shows"]
    store -> stage [label="points to"]
    store -> host [label="serves from"]
}

\newpage
\subsection{Back to Organizations}

Once we have these new models we can assign them to organizations

\digraph[scale=0.80]{OrgModel}{
    splines="ortho";
    rankdir=TD;
    node [shape=box,style=filled,fillcolor="tan"];

    subgraph zero{
        rank=source;
        org [label="Organization"]
    };

    subgraph first{
        scope [label="Scope 1.2"]
    };

    subgraph second{
        rank=same;
        catalog [label="Master Catalog"];
        view [label="Master View"];
    };
    subgraph third{
        view1 [label="View A"];
        view2 [label="View B"];
    };

    subgraph fourth{
        catalog1 [label="Catalog A"];
        catalog2 [label="Catalog B"];
    };

    subgraph fifth{
        channel1 [label="Channel A"];
        channel2 [label="Channel B"];
    };

    subgraph sixth {
        rank=sink;
        store1 [label="Storefront A"];
        store2 [label="Storefront B"];
    };

    org -> scope [label="belongs to"]
    scope -> catalog;
    scope -> view;
    catalog -> catalog1
    catalog -> catalog2
    view -> view1 [label="branch"];
    view -> view2 [label="branch"];
    view1 -> catalog1
    view2 -> catalog2
    catalog1 -> channel1 [label="uses"];
    catalog2 -> channel2 [label="uses"];
    store1 -> channel1 [label="used by"];
    store2 -> channel2 [label="used by"];
}

\subsection {Public Access}

Our current public endpoints for searching the catalog, registering users, and
viewing/purchasing products are not multi-tenant aware at the moment. We need to 
modify them to understand which channel they are serving.


\digraph[scale=0.80]{Proxy}{
    splines="ortho";
    rankdir=TD;
    node [shape=record,style=filled,fillcolor="lightblue"];

    client [label="Client"];
    riverrock [label="River Rock Proxy"];
    db [shape=note,label="DB"];
    channel [label="Channel for host.com"];
    catalog [label="Catalog"];
    view [label="View"];

    client -> riverrock [label="request host.com"]
    riverrock -> db [label="find channel"]
    db -> channel [label="found"];
    channel -> view;
    channel -> catalog;
    view -> riverrock [label="serve"];
    catalog -> riverrock [label="serve"];

}

\subsection{Example Story}

TODO: Give example of single merchant with multi storefronts

\end{document}
