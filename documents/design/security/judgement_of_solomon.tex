\documentclass[11pt]{article}
\usepackage[utf8]{inputenc}
\usepackage[english]{babel}
\usepackage{hyperref}
\usepackage{listings}
%for dotex
\usepackage[pdftex]{graphicx}
\usepackage[pdf]{graphviz}
\usepackage[boxed]{algorithm2e} %end for dote
\usepackage{color}

\title{The Judgement of Solomon}
\begin{document}
\maketitle
\section{The Problem}

There are many services that will manage resources. The challenge is how do we
assign resources to the appropriate users and manage access control
without requiring each service detailed knowledge and logic of the permission
system?

\section{The Solution}

Each service that manages data needs to group data by some mechanism. For our
purposes we will call it a scope. Each request to a service will have a claim
to a scope inside a JWT token. This token will be signed by a service called 
Solomon.

\section{Solomon is Wise}

Solmon's job is to add claims to scopes inside the JWT tokens.
Users who log in will get a JWT token with all the claims they have inside
the JWT token.

Solomon will maintain a mapping table of users to roles and roles to claims. 

Solomon will walk the mapping tables and add the claims to the users JWT, then
sign the JWT and return it.

The user can then pass this JWT token to a downstream service to retrieve the
resources they have claims to.

In addition to creating JWT tokens, Solomon will also provide endpoints to manage
accounts, users, roles, scopes, and organizations.

\section{Many Shepards, Many Sheep}

Since FoxCommerce will support many types of eCommerce systems which may have
varying permissions systems and business logic around access, we want to avoid
building all that logic within Solomon.

We do this by taking the responsibility for maintaining what the different roles,
users, and organizations are created in another service called a Shepard.

For example, a simple one storefront customer will have separate security requirements
than a marketplace. 

We can build out different Shepard services depending on the needs of our customers.

All Solomon can do is understand the mapping table and follow it. It doesn't care
or understand what the roles and claims actually represent. 
It doesn't understand that an organization may be a merchant or vendor, for
example.

\section{Design}
 
\begin{enumerate}
    \item Each system groups resources under scopes.
    \item JWT tokens are used to make claims.
    \item Solomon understands the user to scope, and scope to scope mapping tables.
    \item Shepard services maintain the scope mapping tables.
    \item Isaac validates JWT tokens are authentic.
\end{enumerate}

\digraph[scale=0.80]{SolomonDesign}{
    splines="ortho";
    rankdir=LR;
    node [shape=box,style=filled,fillcolor="lightblue"];

    subgraph zero{
        client [shape=egg, label="Client"]
    };
    subgraph first{
        solomon [shape=record,label="{Isaac|Solomon}"];
        shepard [shape=record,label="{Isaac|Shepard}"];
        DB [shape=box3d, label="Claim DB"];
    };
    subgraph second {
        oms [shape=record,label="{Isaac|OMS}"];
        pim [shape=record,label="{Isaac|PIM}"];
    };

    client -> solomon [dir="both", label="Login returns JWT"];
    solomon -> DB;
    shepard -> solomon;
    client -> oms[label="JWT"];
    client -> pim[label="JWT"];
    client -> shepard[label="JWT"];
}

\section{Scopes}

Scopes are a hierarchical id that is used to group resources. Scopes can have
child scopes.

For example, lets say we have a scope...

\begin{quote}
"1"
\end{quote}

Which represents a tenant. That tenant can add merchants scopes and adds the 
first merchant 

\begin{quote}
"1.1"
\end{quote}

And then second merchant gets

\begin{quote}
"1.2"
\end{quote}

The tenant has de facto access to "1.1" and "1.2" because their scope "1" is a 
prefix match to those other scopes.

Services can do a prefix match on the claimed scope to see if the client
has access to the requested scope.

\section{Claims}

Solomon will add claims to JWT tokens. The claims will be represented as URIs
to the resources under some scope. 

The format of the URI is.

\begin{quote}
    "frn:\textless system\textgreater :\textless resource\textgreater :\textless scope\textgreater "
\end{quote}

Where the order grows in specificity. "frn" stands for "Fox Resource Name"
and is modelled after a URN which is a type of URI.

For example, a claim to orders may look like this.

\begin{quote}
    "frn:oms:order:1.2"
\end{quote}

Where the first part of the URI is the resource "frn:oms:order" and the second
part is the scope "1.2".

A claim in the JWT above is divided into the resource and the action allowed.
Which actions are allowed for a resource depends on the service managing that resource.

An admin user for organization "z" which maps to scope "1.2" might have claims to all 
use orders. 

\begin{lstlisting}
    {
        "frn:oms:order:1.2": ["c", "r", "u", "d"]
    }   
\end{lstlisting}

Where the actions represent create, read, update, and delete.

\section{Use Cases}
\subsection{Assigning Users to Roles}

Solomon's job is to validate claims, however the service is not responsible 
maintaining and augmenting the role mappings. That job is delegated to another
service we shall call the Shepard service.

\subsection{User requests access to a resource from a service}

The user will give the service the JWT token. In front of the service Isaac
will validate the JWT token. The service can then decode the JWT token and 
check that the user has a claim to the requested resource. 

\subsection{User requests a new resource from a service}

Similar to the access request use case, the JWT token is validated by Isaac.
The service can then create the new resource giving it the scope the user requested
as long as they have a claim to data under that scope.

\subsection{Careful Considerations}

\begin{itemize}
    \item How do we put Isaac in front of services, nginx?
    \item Does it make sense the way Isaac validates users and Solomon roles?
\end{itemize}


\end{document}
