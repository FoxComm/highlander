\documentclass[11pt]{article}
\usepackage[utf8]{inputenc}
\usepackage[english]{babel}
\usepackage{hyperref}
\usepackage{listings}
%for dotex
\usepackage[pdftex]{graphicx}
\usepackage[pdf]{graphviz}
\usepackage[boxed]{algorithm2e} %end for dote
\usepackage{color}

% "define" Scala
\lstdefinelanguage{scala}{
  morekeywords={abstract,case,catch,class,def,%
    do,else,extends,false,final,finally,%
    for,if,implicit,import,match,mixin,%
    new,null,object,override,package,%
    private,protected,requires,return,sealed,%
    super,this,throw,trait,true,try,%
    type,val,var,while,with,yield},
  otherkeywords={=>,<-,<\%,<:,>:,\#,@},
  sensitive=true,
  morecomment=[l]{//},
  morecomment=[n]{/*}{*/},
  morestring=[b]",
  morestring=[b]',
  morestring=[b]"""
}

\lstset{ %
    language=scala,
    identifierstyle=\textbf
}

\title{Discount Model}
\author{Maxim \& Pavel}
\begin{document}
\maketitle
\section{Purpose}

Customers want to be able to discount products and run promotions to help drive
sales. Promotions are a merchandising tool and having a flexible one allows the
customer to be creative in their business.

\section{Forces}

\begin{itemize}
    \item Provide flexibility without sacraficing integrity. 
    \item This is a combinatorial problem.
    \item Must only allow valid configurations.
\end{itemize}

\section{Architecture}

\subsection{Parts}
The discounting system has two parts. 

\begin{enumerate}
    \item Merchandising Part
    \item Execution Part
\end{enumerate}

The merchandising part will use our Object System to store information that can
change by context. The execution part will be composed of several pieces.

\begin{itemize}
    \item Qualifier and Offer Algebras
    \item Algebra Compiler
    \item Discount Executor
    \item Promotion Matching
\end{itemize}

\subsection{Flow}

\begin{enumerate}

    \item The promotions/discounts are stored in our DB. 
    \item Given an order, the system will find a promotion.
    \item Take promotion and compile discounts.
    \item Execute discounts on order to produce line item adjustments.
\end{enumerate}

\begin{figure}
\caption{Flow}
\digraph{Flow} {
    node [shape=record];
    Promotion [label="Promotion"];
    Order[label="Order"];
    Execution[label="Executor"];
    Adjustments[label="Line Item Adjustments"];
    Finder[label="Promotion Search"];

    Order -> Finder [label="Input"];
    Finder -> Promotion [label="Produces"];
    Promotion -> Execution[label="Input"];
    Order -> Execution[label="Input"];
    Execution -> Adjustments[label="Produces"];
}
\end{figure}

\begin{figure}
\caption{Applying Discounts}
\digraph{Apply} {
    node [shape=record];
    Discount [label="Discount"];
    Qualifier [label="Qualifier"];
    Offer[label="Offer"];
    CompiledQualifier [label="Compiled Qualifier"];
    CompiledOffer[label="Compiled Offer"];
    Order[label="Order"];
    Compiler[label="Compiler"];
    Execution[label="Executor"];
    Adjustments[label="Line Item Adjustments"];

    Discount -> Qualifier [label="Has"];
    Discount -> Offer [label="Has"];
    Qualifier -> Compiler [label="Input"];
    Offer -> Compiler [label="Input"];
    Compiler -> CompiledQualifier [label="Produces"];
    Compiler -> CompiledOffer [label="Produces"];
    Order -> Execution [label="Input"];
    CompiledQualifier -> Execution [label="Input"];
    CompiledOffer -> Execution [label="Input"];
    Execution -> Adjustments [label="Produces"];
}
\end{figure}
\newpage

\section{Models}

The central concept is the discount. The discounts are bundled in a promotion
and a promotion can be made public via a coupon.

\section{Discounts}

Discounts have three parts.

\begin{enumerate}
    \item Merchandising part.
    \item Qualifier part.
    \item Offer part.
\end{enumerate}

\subsection{Merchandising Discounts}

Each discount will have merchandising data attached to it that can be used for organizing
and merchandising. For example, a title, a description, and tags. The merchandising
information can change based on the context. 

\subsection{Qualifying Discounts}

A discount can only be used by a customer if they qualify for it. The qualification
is going to be expressed as a predicate function which takes a customer/order
bundle as a parameter.

This predicate function will be expressed in a simple algebra which allows
the user to configure the predicate function.

\subsection{Offers}

A discount also has an offer, which describes how the order changes. For example,
"30% off the order" is an offer. 

The offer will be expressed as a function that takes a customer/order bundle as a
parameter and produces line item adjustments.

This offer function will be expressed as a simple algebra which allows the user
to configure the offer to their liking.

\section{Promotions}

Promotions are bundles of discounts and have two parts

\begin{enumerate}
    \item Merchandising part.
    \item Discount bundle.
\end{enumerate}
\subsection{Merchandising Promotions}

Promotions can have merchandising information such as title, description, images,
and tags. This information is used for both display and organization. Merchandising
information will change based on context.

\subsection{Discount Bundle}

The essence of the promotion is that it is composed of one or more discounts bundled
together. This is a simple set of discount references.

\section{Coupons}

Coupons are a way to restrict access to a promotion that isn't available publicly. 
Coupons have three parts.

\begin{enumerate}
    \item Coupon code.
    \item Merchandising part.
    \item Promotion.
\end{enumerate}

\subsection{Coupon code}

The coupon code is a way to provide a handle to a hidden promotion. The code can
be shared to a specific customer, or to many customers. The code is generated
on demand when a coupon is created or copied. We may provide facilities to 
bulk generate codes.

\subsection{Merchandising a Coupon}

Coupons may be displayed to a customer to entice them to use the code. For example
The coupon may be sent via an email and have a title, description, and image.

The merchandising information will change depending on context.

\subsection{Promotion}

The coupon finally points to a specific promotion.

\section{Data Model}

Here we will describe the data model of discounts, promotions, and coupons as
they would appear to a client. The description is in JSON. 

Since the discount, promotion, and coupon may change depending on context, 
we will store the information in our existing Object storage system.
This will give us the power of change  based on context but also versioning.

\subsection{Discount}

\begin{lstlisting}
    //MVP Discount
    {
        title: { type: "string", ...  }
        description: { type: "richText", ...  }
        tags: { type: "tags" ...  }
        qualifier: { type: "qualifier" ...  }
        type: { type: "discountType" ...  }
        offer: { type: "offer" ...  }
    }
\end{lstlisting}

The qualifier and offer will be stored as a JSON Object which represents the AST
of the respective algebra grammars.

\subsection{Promotion}

\begin{lstlisting}
    //MVP Promotion
    {
        title: { type: "string", ref : "..."  }
        description: { type: "richText", ref : "..."  }
        tags: { type: "tags" ref : "..."  }
        images: { type: "images" ref : "..."  }
        activeFrom: { type: "date" ref : "..."  }
        activeTo: { type: "date" ref : "..."  }
        isPublic: { type: "bool" ref : "..."  }
    }
\end{lstlisting}

The set of discounts will be stored using our ObjectLink system.

\section{Coupon}

\begin{lstlisting}
    //MVP Promotion
    {
        code: { type: "string", ref : "..."  }
        title: { type: "richText", ref : "..."  }
        description: { type: "richText", ref : "..."  }
        tags: { type: "tags" ref : "..."  }
        images: { type: "images" ref : "..."  }
        maxUses: { type: "number" ref : "..."  }
        maxCustomerUses: { type: "number" ref : "..."  }
    }
\end{lstlisting}

Additionally, the Coupon in a context will have the following stored

\begin{lstlisting}
    class CouponUsage(id: Int, couponId: Int, total: Int) 
    class CouponCustomerUsage(id: Int, usageId: Int, 
        customerId: Int, total: Int) 
\end{lstlisting}

This is to keep track of how many times it was used in total and by a specific
customer.



\section{Discussion}

\begin{itemize}
    \item Should the coupon merchandising information simply come from the promotion
        instead of storing it at the coupon level?
\end{itemize}
\end{document}
