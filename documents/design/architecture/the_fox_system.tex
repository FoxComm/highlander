\documentclass[11pt]{article}
\usepackage[utf8]{inputenc}
\usepackage[english]{babel}
\usepackage{hyperref}
\usepackage{listings}
%for dotex
\usepackage[pdftex]{graphicx}
\usepackage[pdf]{graphviz}
\usepackage[boxed]{algorithm2e} %end for dote
\usepackage{color}

\title{The FoxCommerce System Architecture}
\begin{document}
\maketitle
\section{Overview}

This article describes various aspects of the FoxCommerce Architecture.
We will go over three parts of the system.

\begin{enumerate}
    \item Overall Design
    \item Scalability
    \item Data Model
\end{enumerate}

\section{Overall Design}
\subsection{Perceptive}

The single best word to describe the FoxCommerce architecture is \emph{Perceptive}. 
Using the Event Sourcing Pattern, the system responds immediately to changes. 
Tasks such as indexing, messaging, alerting, and integrating with 3rd party systems
happen in real time.

The Event Sourcing Pattern allows access to detailed information about the system
enabling insights that would not have been possible otherwise.

\subsection{Event Sourcing}
Most Databases record every change event to a log. The database then applies the change 
events one after another to the data. You can think of the data in the database as a snapshot
of the change events. Event sourcing is the idea of storing
events in a append only data store.

The FoxCommerce system takes the database log and saves it in a distributed messaging system. 
Event Consumers process these change events to do various tasks like indexing, messaging, 
analysis, and alerting, just-in-time as events are received. 
This allows us to avoid batch processing and enables a responsive system.
We call this event sourcing architecture the ``Green River"

\digraph[scale=0.80]{EventSourcing}{
    splines="ortho";
    rankdir=TD;
    node [shape=box,style=filled,fillcolor="lightblue"];

    subgraph zero{
        rank=same;
        App [shape=rectangle,label="API Server"];
    };
    subgraph first{
        rank=same;
        DB [shape=box3d];
        Log [shape=signature];
    };
    subgraph second {
        rank=same;
        Kafka [label="Event Store",shape=note];
        Workers [shape=oval];
    };
    subgraph third {
        rank=same;
        Messaging [shape=tab];
        Analysis [shape=rectangle];
        Alerts [shape=rectangle];
        Search [label="Search Cluster",shape=box3d];
    };
    App -> DB;
    DB -> Log [label="Stores"];
    Log -> Kafka;
    Kafka -> Workers
    Workers -> Search
    Workers -> Messaging
    Workers -> Analysis
    Workers -> Alerts
}

\subsection{Everything Searchable}

Because of the event sourcing architecture, we index all data in a search cluster.
This enables all data in the system to be searchable using a flexible query language.
The search cluster can be used to get data instead of the DB which allows improved
performance and functionality like full text search.

\subsection{Extensible}

There are three ways in which the system is extensible.
\begin{enumerate}
    \item API First
    \item API.js
    \item Event Consumers
\end{enumerate}

\subsubsection{API First}

The system is designed API first. This means that all our components are programmed 
against an HTTP API that is available to everyone. This allows customers to build custom 
applications and other components against the system. The API is divided into public and 
private versions which the customer has full access to.

\digraph[scale=0.80]{Api}{
    splines="ortho";
    rankdir=LR;
    node [shape=box,style=filled,fillcolor="lightblue"];

    subgraph clusterClient{
        label="Clients";
        rank=same;
        client [label="Custom Client",shape=rectangle];
        store [label="Web Storefront",shape=rectangle];
        admin [label="Fox Admin",shape=rectangle];
        mobile [label="Mobile App",shape=rectangle];
    };
    api [shape=rectangle, label="API Servers"];
    client -> api
    store -> api
    admin -> api
    mobile -> api
}

\subsubsection{API.js}

Ease of use is an incredibly important design criteria. That is why we built 
an easy to use JavaScript library to query the API. This library provides functions
to get and put data into the system. The customer can then build their own components
easily on top. The library is designed to use the simplest and most
straightforward JavaScript.

\subsubsection{Event Consumers}

Because of the Event Sourcing architecture, customers can also build custom
event consumers that respond immediately to changes in the database. This can 
greatly simply integration and synchronization with 3rd party systems. For example
synchronizing orders with an ERP and 3PL. 

\digraph[scale=0.80]{EventListen}{
    rankdir=LR;
    node [shape=box,style=filled,fillcolor="lightblue"];

    store [shape=note, label="Event Store"];
    subgraph clusterWorkers{
        label="Event Consumers";
        rank=same;
        idx [label="Fox Search Indexing"]
        msg [label="Fox Messaging"]
        erp [label="ERP Sync"]
        tpl [label="WMS Sync"]
    };
    store -> idx
    store -> msg
    store -> erp
    store -> tpl
}


\subsection{Activity Trail}

In addition to monitoring changes in the database. The API Servers generate
high level events called Activities. These Activities are linked to Activity Trails that
belong to various objects in the system. You can monitor high level changes to all parts
of the system including customers, orders, admins and products. You can see the same activity in 
different timelines. For example, if a customer changes the shipping address on their order,
you will see the activity on the order's and customer's trails.

\digraph[scale=0.80]{Insights}{
    rankdir=TD;
    node [shape=box,style=filled,fillcolor="lightblue"];

    subgraph clusterApp{
        label="Activites Generated";
        app [shape=note, label="App"];
        act1 [shape=note, label="activity 1"];
        act2 [shape=note, label="activity 2"];
        act3 [shape=note, label="activity 3"];
    };

    subgraph clusterCustomer{
        label="Customer Trail";
        rank=same;
        cact1 [shape=note, label="activity 1"];
        cact2 [shape=note, label="activity 2"];
    };

    subgraph clusterOrder{
        label="Order Trail";
        rank=same;
        oact2 [shape=note, label="activity 2"];
        oact3 [shape=note, label="activity 3"];
    };

    subgraph clusterAdmin{
        label="Admin Trail";
        rank=same;
        aact2 [shape=note, label="activity 2"];
    };

    app -> act3
    act3 -> act2
    act2 -> act1
    act1 -> cact1
    cact1 -> cact2
    act1 -> oact2
    oact2 -> oact3
    act1 -> aact2
}

\section{Scalability}

Running a production system can be a significant cost. It is important that the
system is designed to be the right size at the right time. The architecture needs
to be designed so that the system can scale up and down without problems. 

We designed the system to be scalable in several ways.

\begin{enumerate}
    \item Stateless
    \item Data Partitioning
    \item Disjoint Parallelism
\end{enumerate}

\subsection{Stateless}

A very important property of scalable systems is statelessness. 
The backend software is written in Scala and we don't allow mutable state to 
be stored within the program. All software serving requests to clients is stateless 
which allows nodes to be added or removed without affecting behaviour of the system. 

This allows us to easily add a load balancer and then auto scale the machines based
on utilization. We use few machines during slow times and scale up resources
as demand increases. This means an overall cheaper system because there is less waste.

\digraph[scale=0.80]{Stateless}{
    rankdir=TD;
    node [shape=box,style=filled,fillcolor="lightblue"];

    client [shape=note, label="Client"];
    balancer [shape=note, label="balancer"];
    subgraph clusterWorkers{
        label="Stateless Nodes";
        rank=same;
        n1 [label="Node"]
        n2 [label="Node"]
        n3 [label="Node"]
        n4 [label="Node"]
    };

    client -> balancer
    balancer -> n1
    balancer -> n2
    balancer -> n3
    balancer -> n4
}

\subsection{Data Partitioning}

The data in the system is always growing and therefore it is important to partition
the data. The FoxCommerce system has several stores of data and each of these 
is designed to be partitioned.

\begin{enumerate}
    \item DB Sharding.
    \item Event Store Partitioning.
    \item Search Cluster Index Partitioning.
\end{enumerate}

Each of these is partitioned so that as the data grows linearly, the query 
times stay consistent.

\subsection{Read/Write Partitioning}

Another important form of partitioning is separating where data is written to and
where it is read from. Because data is indexed as soon as it changes, much of the frontend UI code 
reads from the search cluster instead of the DB. This means that the DB has less
work to do which improves overall system performance.

\subsection{Disjoint Parallelism}

Just as the API servers nodes can be added or removed, the backend processing
components need be added and removed without affecting system performance. 

The Event Store is partitioned so that you can have many event consumers working
in parallel without affecting each other. For example, the indexing and messaging
can process the same event stream at the same time.

\digraph[scale=0.80]{Disjoint}{
    rankdir=LR;
    node [shape=box,style=filled,fillcolor="lightblue"];

    store [shape=note, label="Event Store"];
    subgraph clusterWorkers{
        label="Parallel Event Consumers";
        rank=same;
        idx [label="Search Indexing"]
        msg [label="Messaging"]
        elp [label="..."]
    };
    store -> idx
    store -> msg
    store -> elp
}

\section{Data Model}

An important technique in creating a good system architecture and data model
is understanding the symmetry in the system and where the symmetry breaks. Where
symmetry breaks is where there is something interesting happening. During the Big Bang,
there was equal parts matter and dark matter. The symmetry of the universe somehow
broke and all the dark matter went somewhere. Broken symmetry is why we are here today. 

The FoxCommerce data model has a broken symmetry and is divided into two parts.

\begin{enumerate}
    \item Transactions
    \item PIM/Merchandising 
\end{enumerate}

\subsection{Transactions}

Transactions are core to the business and it is important that the data is 
correct and the code is safe. Several techniques are used to maintain correctness
and safety.

\begin{enumerate}
    \item Strong Type System
    \item Finite State Machines
    \item Referential Integrity
\end{enumerate}

The backend is written in Scala. For the transactional part of the system, 
the strength of Scala's type system is utilized to validate correctness.
It is important that the transactional part of the system is consistent and that
the data is always in a correct state. Strongly typed data structures
are used to represent Orders and the possible states they can have. 

Orders are modeled as a Finite State Machine where the valid state transitions
are checked at compile time. We use referential integrity in our database
model to make sure everything is structured correctly and that all relations
are maintained. Pointing to non-existent data is not possible.

\digraph[scale=0.80]{OrderFSM} {
    splines="ortho";
    node [shape=rectangle,style=filled,fillcolor="lightblue"];
    subgraph clusterFSM {
        order [label="Order"];
        label="Finite State Machine"
        cart [label="Cart"];
        fraud [label="Fraud Hold"];
        remorse [label="Remorse Hold"];
        manual [label="Manual Hold"];
        full [label="Fullfillment Started"];
        shipped [label="Shipped"];
        canceled [label="Canceled"];
    };
    order -> cart
    cart -> fraud
    cart -> remorse
    cart -> canceled
    fraud -> manual
    fraud -> remorse
    fraud -> canceled
    remorse -> fraud
    remorse -> manual
    remorse -> full
    remorse -> canceled
    manual -> fraud
    manual -> remorse
    manual -> full
    manual -> canceled
    full -> shipped
    full -> canceled
}

\subsection{PIM/Merchandising}

While the transactions data model is built using principles of correctness and 
safety, the product and merchandising model is designed to be maximally flexible.

The PIM/Merchandising data model has three important components

\begin{enumerate}
    \item Extensible Object Model
    \item Versioning
    \item Taxonomy
\end{enumerate}

\subsubsection{Extensible Object Model}

The foundation of the product and merchandising model is a flexible object store
where objects can have custom properties and dynamic associations. These objects
are viewed from something we call a ``Context". The Context is the lens you
view the object with.

For example, a product might be viewable in a USA and Russia context. When you
change the USA context, you see the product's USA specific properties, like an English
description. When you switch to the Russian context, you would see the Russian description,
and any other additional properties specific to Russia. The same object lives
in both contexts. We call this model the Form/Shadow model.

\digraph[scale=0.80]{Shadow} {
    splines="ortho";
    node [shape=record,style=filled,fillcolor="lightblue"];
    Context [label="Context"];
    Form [label="Product Form"];
    Shadow1 [label="USA Shadow"];
    Shadow2 [label="Russia Shadow"];
    Shadow3 [label="Mobile Shadow ..."];

    Context -> Form [label="Illuminates"];
    Form -> Shadow1 [label="Projects"];
    Form -> Shadow2 [label="Projects"];
    Form -> Shadow3 [label="Projects"];
}

This model allows you to sanely manage merchandising various parts of the
system to different groups of people.

\subsubsection{Versioning}

In addition, each change to any object in the merchandising system is versioned.

The versioning model is built on the Form/Shadow and Context model. 

\digraph[scale=0.8]{ProductVersion} {
    node [shape=rectangle,style=filled,fillcolor="lightblue"];
    Context [label="Context"];
    {
        rank=same;
        Head1[label="..."];
        Head2 [label="..."];
        Head3 [label="Head"];
    }
    {
        rank=same;
        Commit1 [label="Commit 1"]
        Commit2 [label="Commit 2"]
        Commit3 [label="Commit 3"]
    }
    {
        rank=same;
        Shadow1 [label="Shadow 1"];
        Shadow2 [label="Shadow 2"];
        Shadow3 [label="Shadow 3"];
    }
    Form [label="Product Form"];

    Context -> Head1 [label="Has"];
    Context -> Head2 [label="Has"];
    Context -> Head3 [label="Has"];

    Head3 -> Commit3 [label="Points to"];
    Commit1 -> Commit2[dir="back"];
    Commit2 -> Commit3[dir="back"];
    Commit1 -> Shadow1;
    Commit2 -> Shadow2;
    Commit3 -> Shadow3;
    Shadow1 -> Form 
    Shadow2 -> Form 
    Shadow3 -> Form [label="Snapshot of"]
}

It allows the same object to have different versions in both time and context.
This can enable very interesting workflows. For example, if every Admin
had their own Context, they can make changes to a product and have the changes
reviewed before they go live.

It can allow whole changes to a site while creating a holiday campaign.
You can have the campaign go live and then switch back to the older version in
the previous context.

\subsubsection{Taxonomy}

There is also a flexible model for organizing the objects. An object can be assigned
various to taxonomies like categories and tags. These can be used to organize objects
into groups and trees.

Since all data is indexed, groups can be defined as queries on the taxonomies and
properties of objects. This allows you to have dynamic groups which change over time.

For example, you can have a group ``top 10 popular sun glasses", and have it
defined as a search query based on sales and taxonomy.

\end{document}
