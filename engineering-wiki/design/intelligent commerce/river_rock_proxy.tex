\documentclass[11pt]{article}
\usepackage[utf8]{inputenc}
\usepackage[english]{babel}
\usepackage{hyperref}
\usepackage{listings}
%for dotex
\usepackage[pdftex]{graphicx}
\usepackage[pdf]{graphviz}
\usepackage[boxed]{algorithm2e} %end for dote
\usepackage{color}

\title{River Rock Proxy: The Green River Polishes Stones}
\author{Maxim Noah Khailo}
\begin{document}
\maketitle
\section{Intelligent Digital Channels}

This paper will focus primarily on building intelligent digital channels.

\subsection{Proposition}

Imagine a system where you provide product data, layouts, colors, imagery and flows 
with possible variations and the system can figure out over time which is the most appropriate
at any given moment. Like a river polishing stones.

\section{Polishing the Digital 3rd Place}

A virtual storefront is very different from a physical one. It has it's 
advantages and disadvantages and we shall soon see that it's advantages can 
become disadvantages in disguise. 

With a physical store you can readily try different arrangements of the space and
witness customer reaction. You can improve the products and the space
incrementally over time until you get something customers love and feel good about.

Paradoxically, a virtual storefront is often much harder to rearrange and incrementally
improve. Typical E-commerce systems provide very rigid structures like website
templates which take great expertise to improve and change. 

Instead of picking up a shoe and moving it to the other end of a store, you have 
to pay someone with the expertise to change an HTML template or tweak CSS style sheets.
If you want more drastic changes like the way your customers buy from your storefront,
you can bet it will require months of effort.

One primary advantage that a virtual storefront has is that it is infinitely large. Any
amount of people can visit it and browse it. This same advantage becomes a massive
disadvantage since most signal gets lost in the noise.

It becomes very difficult to understand who the customers are and how they are reacting
to your changes. You need specialized tools for analysis to start to grasp
what is occurring. You can't see a person's reaction and watch how they browse as you did with
a physical store. And even when you have a grasp, deciding what to do is difficult.

Another advantage of a virtual storefront that becomes a disadvantage 
is the amount of possible improvements and arrangements is much larger than
a physical space. While changing the color of a button on a website is cheaper
than painting a wall, the possible colors, shapes, locations of a button are large.

Combining the increase in potential customers, an increase in possible arrangements, and a
decrease in insight and observability, produces a huge combinatorial and optimization problem.

This problem can no longer be solved without automated assistance.

\section{The River Rock Proxy Overview}

The River Rock Proxy 

\end{document}
