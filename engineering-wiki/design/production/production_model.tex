\documentclass[11pt]{article}
\usepackage[utf8]{inputenc}
\usepackage[english]{babel}
\usepackage{hyperref}
\usepackage{listings}
%for dotex
\usepackage[pdftex]{graphicx}
\usepackage[pdf]{graphviz}
\usepackage[boxed]{algorithm2e} %end for dote
\usepackage{color}

% "define" Scala
\lstdefinelanguage{scala}{
  morekeywords={abstract,case,catch,class,def,%
    do,else,extends,false,final,finally,%
    for,if,implicit,import,match,mixin,%
    new,null,object,override,package,%
    private,protected,requires,return,sealed,%
    super,this,throw,trait,true,try,%
    type,val,var,while,with,yield},
  otherkeywords={=>,<-,<\%,<:,>:,\#,@},
  sensitive=true,
  morecomment=[l]{//},
  morecomment=[n]{/*}{*/},
  morestring=[b]",
  morestring=[b]',
  morestring=[b]"""
}

\lstset{ %
    language=scala,
    identifierstyle=\textbf
}

\title{The Happy Production Model}
\author{Maxim Noah Khailo}
\begin{document}
\maketitle
\section{Purpose}

The best production model is a happy one. Software is not some ephemeral fairy that
lives in a world outside out own. It has a real physical nature. When you build
a software system, you are building a machine, with moving parts. 

However, machines tend to break down and need constant maintenance. This makes
engineers unhappy. So we must resurrect the echo of Alan Kay to flip our idea
of software running on a machine on it's head. There are other machines that are
more resilient, that have maintenance built in as automatic systems. These are
biological in nature. So we should look to nature for a system that is more resilient
against entropy. 

A happy production model is built on one that models more of a biological system
than a mechanical one.

\section{Properties of Biological systems}

Biological systems have several properties we should emulate.

\begin{enumerate}
    \item Redundancy
    \item Communication
    \item Decentralization with Centralization
    \item Scale
    \item Self Healing
\end{enumerate}

Redundancy is necessary for resilience because we want to eliminate any single
point of failure.  

Biological systems are composed of trillions of self contained systems called cells. 
These cells are encapsulated in a membrane and achieve coordination via communication. 
True object oriented programming involves communication between components. In 
software systems, the communication is usually achieved via network protocols. 

Higher level biological systems have a very interesting property. Despite being composed
of trillions of decentralized communicating pieces, they also have central nervous
systems helping coordinate functions. This is no accident. Performance increases 
for decentralized software systems when a small amount of centralization is introduced. 

Biological systems are designed to grow. Different parts grow at different 
rates and at different times. For our production system, we will have to make it 
clear where the software can grow and by which stimuli. 

Our bodies have agents that constantly clean the system. Our skin heals when you cut it.
The fight for life is the fight against entropy. Similarly software systems 
need constant monitoring and healing. Machines and services that are not 
communicating anymore need to be automatically killed and replaced.

\section{MVP Production Model}

Given the properties outlined above, what would a minimal FoxCommerce production
system look like?

\digraph[scale=0.50]{MVPModel}{
    splines="ortho";
    rankdir=LR;
    subgraph cluster_region1 {
        label="Region 1";
        subgraph cluster_kafka1 {
            label="Kafka";
            K1 [label="Cluster..."];
        };
        subgraph cluster_green1 {
            label="Green River";
            G1 [label="Workers..."];
        };
        subgraph cluster_elastic1 {
            label="Elastic Search";
            E1 [label="Cluster..."]
        };
        subgraph cluster_frontend1 {
            label="Frontend";
            F1 [label="Phoenix Workers..."]
            A1 [label="Node Workers..."]
        };
        subgraph cluster_db1 {
            label="Db";
            Db [label="Master"];
            Db -> K1;
            K1 -> G1;
            G1 -> E1;
            F1 -> Db [dir=both, label="RW"];
            A1 -> E1 [dir=back, constraint=false];
        };
        subgraph cluster_sd1 {
            label="Service Discovery";
            C1 [label="Consul..."];
            Z1 [label="Zookeeper..."];
        };
    }

    subgraph cluster_region2 {
        label="Region 2";
        subgraph cluster_kafka2 {
            label="Kafka";
            K2 [label="Cluster..."];
        };
        subgraph cluster_green2 {
            label="Green River";
            G2 [label="Workers..."];
        };
        subgraph cluster_elastic2 {
            label="Elastic Search";
            E2 [label="Cluster..."]
        };
        subgraph cluster_frontend2 {
            label="Frontend";
            F2 [label="Phoenix Workers..."]
            A2 [label="Node Workers..."]
        };
        subgraph cluster_db2 {
            label="Db";
            Db2 [label="Replicas..."];
            Db -> Db2 [label="replicate", constraint=false];
            Db2 -> K2;
            K2 -> G2;
            G2 -> E2;
            F2 -> Db [dir=both, label="RW"];
            A2 -> E2 [dir=back, constraint=false];
        };
        subgraph cluster_sd2 {
            label="Service Discovery";
            Z2 [label="Zookeeper..."];
            C2 [label="Consul..."];
        };
    }
}

\subsection{Two Regions and DB Asymmetry}

The first obvious feature of the architecture is that the system is essentially mirrored
in two regions. Both regions will receive traffic to the frontend instances. 

The system symmetry breaks at the DB level. There will be one master DB that both
regions write two. This master will have replicas in both regions where changes
are streamed using the write ahead log.

When one region goes down, the frontend instances should seamlessly switch over
to the failover master in the other region. There is a mall danger of data loss if 
there is a large delta in time between replication. We can alleviate that by 
requiring writes to only succeed if the replication is successful.

\subsection{Cluster Separation}

Since each region is a copy of the other, we will keep them isolated except for
the DB read/writes and replication. 

This means we will not cluster Kafka, Green River, or ElasticSearch across the regions.
Since all data comes from the DB, the Green River workers should process the same
data and ElasticSearch should have the same index.

Given this, the front end machines can access their respective ES clusters in their 
region. This should help provide nice scalability and load distribution across regions.

\section{Scalability of Components}

\subsubsection{DB}

TBD - Sharding? Someone needs to do research here.

\subsubsection{Kafka}

We will create the appropriate amount of partitions per topic depending on the
topic size. If the topic is large and the messages independent then we can 
partition the topic into N partitions and have N Green River workers processing 
that topic.

Unfortunately we cannot autoscale Kafka and need to be thoughtful in configuring
the topics and partitions. Should we need to add brokers, we would need tell Kafka
which partitions to move over to the new machines.

\subsubsection{ElasticSearch}

Elastic Search has a master/replica setup and does not autoscale well. Similarly
to Kafka, to load balance the data, re-sharding can be done but it is an 
expensive operation and requires re-indexing.

\subsubsection{Phoenix and Ashes/Firebird}

Since both Phoenix and Ashes/Firebird are stateless services, autoscaling them and putting
them behind a load balancer is fairly trivial.

We can decide to separately scale Phoenix and Ashes/Firebird or we can put them
on the same machine and scale them together. 

The advantage of putting them on the same machine is reduced latency. The disadvantage
is we would require more DB connections than necessary.

\section{AWS System Cost Estimate}

The cost described below is a range and uses AWS EC2 prices. I used the standard
hour prices here for instance types. I have also not factored in storage costs like S3,
snapshots, and other backups. Typically storage is the cheapest part of the system.

The total prices below are per month for one region The 2xRegion cost is also listed.

\subsection{Medium}
\begin{center}
    \begin{tabular}{| l l l l | l | p{3cm} |}
        \hline
        System & Machine & Count & 1xPrice & Total/Month & Notes \\
        \hline \hline
        Consul/Zookeeper & t2.small & 3 & \$18.72 & \$56.16 & \\ \hline
        DB & i2.2xlarge & 1 & \$1227.60 & \$1227.60 & The other region will be replica \\ \hline
        Kafka & d2.xlarge & 2 & \$496.80 & \$993.60 & Not using SSD here since access is linear.\\ \hline
        ElasticSearch & r3.xlarge & 3 & \$239.76 & \$719.28 & 1 Master, 2 replicas\\ \hline
        ES Logging & r3.xlarge & 1 & \$239.76 & \$239.76 & 1 Master, S3 backups\\ \hline
        Green River & m3.xlarge & 1 & \$191.52 & \$191.52 & Multi workers on one instance.\\ \hline
        Phoenix/Ashes & m3.xlarge & 1-4 & \$191.52 & \$191.52-\$766.08 & Scales based on traffic.\\ \hline
        & & & total & {\bf \$3618.44-\$4193.00} & \\ \hline
        & & & 2xRegion & {\bf \$7236.88-\$8386.00} & \\ \hline
    \end{tabular}
\end{center}

\subsection{Large}

Notes are based on changes from the medium install.

\begin{center}
    \begin{tabular}{| l l l l | l | p{3cm} |}
        \hline
        System & Machine & Count & 1xPrice & Total/Month & Notes \\
        \hline \hline
        Consul/Zookeeper & t2.medium & 3 & \$37.44 & \$112.32 & Slightly better machines.\\ \hline
        DB & i2.4xlarge & 2 & \$2455.10 & \$4910.20 & One replica per region.\\ \hline
        Kafka & d2.xlarge & 3 & \$496.80 & \$1490.40 & More brokers.\\ \hline
        ElasticSearch & r3.2xlarge & 3 & \$478.80 & \$1436.40 & More RAM.\\ \hline
        ES Logging & r3.2xlarge & 3 & \$478.80 & \$1436.40 & More RAM.\\ \hline
        Green River & m3.2xlarge & 1 & \$383.04 & \$383.04 & More CPU.\\ \hline
        Phoenix/Ashes & m3.2xlarge & 1-4 & \$383.04 & \$383.04-\$1532.16 & More CPU.\\ \hline
        & & & total & {\bf \$10152-\$11301.12} & \\ \hline
        & & & 2xRegion & {\bf \$20304-\$22602.24} & \\ \hline
    \end{tabular}
\end{center}

\subsection{Small}

Notes are based on changes from the medium install.

\begin{center}
    \begin{tabular}{| l l l l | l | p{3cm} |}
        \hline
        System & Machine & Count & 1xPrice & Total/Month & Notes \\
        \hline \hline
        Consul/Zookeeper & t2.small & 3 & \$18.72 & \$56.16 & No Change\\ \hline
        DB & d2.xlarge & 1 & \$496.80 & \$496.80 & Don't use SSD\\ \hline
        Kafka & d2.xlarge & 1 & \$496.80 & \$496.80 & One Machine.\\ \hline
        ElasticSearch & r3.large & 1 & \$119.52 & \$119.52 & One Machine.\\ \hline
        ES Logging & r3.large & 1 & \$119.52 & \$119.52 & One Machine.\\ \hline
        Green River & m3.large & 1 & \$95.76 & \$95.76 & Smaller Machine.\\ \hline
        Phoenix/Ashes & m3.large & 1-4 & \$95.76 & \$95.76-\$383.04 & Smaller Machine.\\ \hline
        & & & total & {\bf \$1480.32-\$1767.60} & \\ \hline
        & & & 2xRegion & {\bf \$2960.64-\$3535.20} & \\ \hline
    \end{tabular}
\end{center}

\section{AWS Cost Optimization}

The minimum cost described above can actually be smaller if we decide to save in a couple of ways. 
We can save in EC2 costs by purchasing reserved instances. Typically discounts can be
up to 50\% depending on the license terms we choose and how much we pay
up front. We should investigate if we can use the AWS credits we have towards
buying reserved instances.

Another cost savings approach we can take is to use spot instances for our stateless
services. AWS has a market for purchasing unused EC2 capacity at a steep discount.
The prices are pretty volatile but the discounts can be steep, up to 80\% off.

The downside of spot instances is that they can be killed at any time should should
the spot price rise above the price you paid. I think spot instances could be 
appropriate for scaling Ashes and Phoenix instances.


\end{document}
