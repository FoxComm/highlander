\documentclass[11pt]{article}
\usepackage[utf8]{inputenc}
\usepackage[english]{babel}
\usepackage{hyperref}
\usepackage{listings}
%for dotex
\usepackage[pdftex]{graphicx}
\usepackage[pdf]{graphviz}
\usepackage[boxed]{algorithm2e} %end for dote
\usepackage{color}

% For use case env, original source found here.
% http://tex.stackexchange.com/questions/10293/latex-template-for-use-cases#10325
\newcommand\tabularhead[1]{
\begin{table}[h]
    \caption{Use Case "#1"}
  \begin{tabular}{|p{0.5\linewidth}|p{0.66\linewidth}|}
    \hline
    \textbf{Action} & \textbf{#1} \\
    \hline}

  \newcommand\addrow[2]{#1 &#2\\ \hline}

  \newcommand\addmulrow[2]{ \begin{minipage}[t][][t]{2.5cm}#1\end{minipage}% 
     &\begin{minipage}[t][][t]{8cm}
      \begin{enumerate} #2   \end{enumerate}
      \end{minipage}\\ }

  \newenvironment{usecase}{\tabularhead}
{\hline\end{tabular}\end{table}}

\title{Multi-Tenant Settings}
\author{Maxim Noah Khailo}
\begin{document}
\maketitle
\section{Problem}

Currently many settings specific to a merchant such as their Stripe , Mailchimp, and
shipstation keys are stored as environment variables to processes running in
marathon. As we go multi-tenant this is clearly not acceptable. 

\section{Outline}

This document provides use cases for implementing mult-tenant settings. Each
use case is briefly written describing the Actor, pre-conditions, post-conditions,
the main path and alternate paths.

This document does not describe how the system should look but describes what it
should be capable of doing.

\section{Use Cases}

\subsection{Onboarding}

New Tenants will go through an onboarding flow allowing then to configure the 
baseline system and 3rd party integrations

\begin{usecase}{Selecting 3rd Party Integrations}
    \addrow{Actor}{Tenant Admin}
    \addrow{Precondition}{Tenant has create an organization and has a master tenant admin account. Tenant Admin is logged in.}
    \addrow{Postcondition}{System will know which 3RD party integrations they want configured}
    \addrow{Main Path}{Admin selects from a list of available 3rd party integrations}
    \addrow{Alternate Path}{No integrations are selected. User goes strait to Admin interface.}
\end{usecase}

\begin{usecase}{Configuring Settings For an Integration}
    \addrow{Actor}{Tenant Admin}
    \addrow{Precondition}{Admin has selected the integrations they want to configure.}
    \addrow{Postcondition}{The current integration in process is configured}
    \addmulrow{Main Path}{
        \item The available settings for the particular integration are displayed.
        \item The appropriate GUI widget is shown for the setting type.
        \item The user fills out all required fields.
        \item The user presses the ``Test" button.
        \item Only with a good test can they continue to next integration.
    }
    \addrow{Alternate Path}{User selects ``Skip". Integration is marked as inactive 
    but available to edit in the future}
\end{usecase}

\subsection{Setting Modification}

Tenants should be able to go back through all integrations in ashes and modify 
existing settings. We also need general Fox commerce related settings available.

\begin{usecase}{Editing Existing Settings}
    \addrow{Actor}{Tenant Admin}
    \addrow{Precondition}{Tenant is logged into Ashes and has permissions to edit settings}
    \addrow{Postcondition}{Settings are modified and available immediately}
    \addmulrow{Main Path}{
        \item Admin can go to settings tab and see settings arranged in some grouping.
        \item The groups they see are only those they are allowed to edit.
        \item They click on a group and can see each individual settings.
        \item Individual settings are displayed same way as during onboarding.
        \item If the settings are an integration, they press the ``Test" button.
        \item Only if ``Test"" succeeds can they save the changes.
        \item If the settings are an integration, they can disable it.
        \item If the setting group is not an integration, Then saving button is
              enabled.
     }
    \addrow{Alternate Path}{They don't want to change any settings, they can leave the screen}
\end{usecase}

\begin{usecase}{Adding an Integration}
    \addrow{Actor}{Tenant Admin}
    \addrow{Precondition}{Tenant is logged into Ashes and has permissions to add integrations}
    \addrow{Postcondition}{Integration is added if successfully goes through flow}
    \addmulrow{Main Path}{
        \item Admin can go to settings tab and see settings arranged in some grouping.
        \item They click on button to add an integration.
        \item They are presented with a list of integrations not already added.
        \item They can then go through the ``Configuring Settings For an Integration" use case.
     }
    \addrow{Alternate Path}{They don't want to add an integration, they can leave the screen}
\end{usecase}

\begin{usecase}{Adding a Settings Group}
    \addrow{Actor}{Tenant Admin}
    \addrow{Precondition}{Tenant has login credentials and has permissions to add a settings group}
    \addrow{Postcondition}{Settings group is added and is available for other admins to edit.}
    \addmulrow{Main Path}{
        \item Admin can add a settings group at the tenant level using an API
        \item They can define setting keys and their types. Types influence how
              the settings are represented in the UI.
        \item They can do CRUD operations on the group via the API.
     }
    \addrow{Alternate Path}{The setting group already exists, they can instead use CRUD API to modify.}
\end{usecase}

\subsection{User Settings}

\begin{usecase}{User Edits Their Settings}
    \addrow{Actor}{Admin}
    \addrow{Precondition}{Admin is logged in and has permission to see their own settings}
    \addrow{Postcondition}{Settings are modified}
    \addmulrow{Main Path}{
        \item Admin can go to settings tab and see settings arranged in some grouping.
        \item The first settings groups are settings specific to them.
        \item They can modify any settings in their own groups.
        \item In addition to their settings, if they have permission they can modify 
            tenant and integration settings.
     }
    \addrow{Alternate Path}{They don't want to change any settings, they can leave the screen}
\end{usecase}

\end{document}
