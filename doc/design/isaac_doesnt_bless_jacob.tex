\documentclass[11pt]{article}
\usepackage[utf8]{inputenc}
\usepackage[english]{babel}
\usepackage{hyperref}
\usepackage{listings}
%for dotex
\usepackage[pdftex]{graphicx}
\usepackage[pdf]{graphviz}
\usepackage[boxed]{algorithm2e} %end for dote
\usepackage{color}

\title{Isaac Doesn't Bless Jacob}
\begin{document}
\maketitle
\section{The Problem}

We need to authenticate users across various systems, both internal and external.
Isaac is a service that validates JWT tokens based on our user database, while
still avoiding sessions.

The use cases this service is designed to address are.
\begin{enumerate}
    \item Validate token signatures.
    \item Invalidate tokens for non existent or banned users.
    \item Invalidate tokens from valid users that were compromised. 
\end{enumerate}

\section{Use Cases}
\subsection{Validate Token Signatures}

The primary purpose of this service is to identify good tokens from bad. The first
line of defense against forged tokens is validating JWT signature.

A JWT token is composed of three parts, the header, payload, and signature.
We are using RS256 signing algorithm which uses asymmetric RSA keys. The JWT
is signed with the private key and validated with the public key.

Since we are only supporting RS256, the implementation is straight forward and
simpler. Simpler is better when it comes to security. Restricting to these kinds
of keys will prevent attacks such as forging a JWT token using HMAC and using
the public key as the secret.

\subsection{Invalidate Tokens for Non Existent or Banned Users}

JWT is designed to be stateless. Meaning that the server does not store any 
session information and the client has the sole responsibility of remembering the
token. However, the only feature of JWT provided for invalidating tokens is an expiration
date. 

We want to support invalidating tokens based on banning and removing users so
we can protect our client's data should we have to.

This is why the service will check the user database to validate whether the
user is active and able to have access to the system.

\subsection{Invalidate Tokens from Valid Users That Were Compromised}

Another challenge is how to invalidate JWT tokens from a valid user but their
token's were compromised some way. Each user will have something called a "ratchet"
which is an integer that can increment to invalidate old tokens.

The service will check and make sure the JWT's ratchet number matches that of
the user in the database.

\section{Design}

\subsection{Overview}

The two biggest attributes the service must have is safety and performance.

Safety because all requests to private routes need to be authenticated. We want
to maintain JWT's sessionless design while still retaining the ability to invalidate
tokens.

Performance because all requests to internal systems will require validation.
The service will have to handle a high load.

To maintain performance, we will utilize a cache to limit DB access and to 
maintain safety, we will use "green river" to prevent a stale cache.

\digraph[scale=0.80]{IsaacDesign}{
    splines="ortho";
    rankdir=TD;
    node [shape=box,style=filled,fillcolor="lightblue"];

    subgraph zero{
        rank=same;
        Client [shape=egg]
    };
    subgraph first{
        rank=same;
        query [shape=rectangle,label="Query Service"];
        cache [shape=signature,label="User Cache Service"];
    };
    subgraph second {
        rank=same;
        DB [shape=box3d];
        green [shape=rectangle,label="Green River"];
    };

    Client -> query;
    query -> Client;
    query -> cache;
    query -> DB 
    DB -> green;
    green -> cache;
}

\subsection{Safety and Performance}

Safety and performance are forces that are constantly pulling on each other.
The main feature for performance in this services is a cache so that the service
can minimize DB access. The problem with maintaining a cache is obviously that
the data goes scale. 

For security, we need to be able to invalidate the cache for a user/token set
as soon as the DB data changes.

Fortunately one foundational feature of the FoxCommerce system is the event sourcing
model. All changes to the DB are handles by "green river" workers. We can then
utilize this system to invalidate the cache in the auth service as soon as data
in the DB changes.

This means we can maintain invalidate JWT tokens with little latency to maintain
safety and prevent a stale cache.

\end{document}
