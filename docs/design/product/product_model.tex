\documentclass[11pt]{article}
\usepackage[utf8]{inputenc}
\usepackage[english]{babel}
\usepackage{hyperref}
\usepackage{listings}
%for dotex
\usepackage[pdftex]{graphicx}
\usepackage[pdf]{graphviz}
\usepackage[boxed]{algorithm2e} %end for dote
\usepackage{color}

% "define" Scala
\lstdefinelanguage{scala}{
  morekeywords={abstract,case,catch,class,def,%
    do,else,extends,false,final,finally,%
    for,if,implicit,import,match,mixin,%
    new,null,object,override,package,%
    private,protected,requires,return,sealed,%
    super,this,throw,trait,true,try,%
    type,val,var,while,with,yield},
  otherkeywords={=>,<-,<\%,<:,>:,\#,@},
  sensitive=true,
  morecomment=[l]{//},
  morecomment=[n]{/*}{*/},
  morestring=[b]",
  morestring=[b]',
  morestring=[b]"""
}

\lstset{ %
    language=scala,
    identifierstyle=\textbf
}

\title{The Enlightened, Post-Modern Product Model}
\author{Maxim Noah Khailo}
\begin{document}
\maketitle
\section{Purpose}

There is both a universal objective reality and our subjective interpretations of 
it. The post-modern world is one of flux and uncertainty which can only collapse
into the shadow of objectivity when illuminated. 

Below is a description of a product model, nay, a content model that both provides
variation in the face of contextualization, but has its objective reality and definition.

Please read the companion paper "Everything is Search" after this one to realize
the full potential of the approach described below.

\section{Forces}

We have to balance several problems.

\begin{itemize}
    \item Provide the most appropriate information at the right time.
    \item Semantically structure information into a computable form.
    \item Provide a manageable way to handle variation.
\end{itemize}

\newpage
\section{The Form and Shadow}

The model is split into two concepts, the \emph{Form} and the \emph{Shadow}. 
A product takes on one form but can have many shadows. Shadows are created when
a \emph{Context} illuminates the \emph{Form}.

\digraph{Parts} {
    node [shape=record];
    Context [label="Context"];
    Form [label="Product Form"];
    Shadow1 [label="Desktop Shadow"];
    Shadow2 [label="Mobile Shadow"];
    Shadow3 [label="Shadow ..."];

    Context -> Form [label="Illuminates"];
    Form -> Shadow1 [label="Projects"];
    Form -> Shadow2 [label="Projects"];
    Form -> Shadow3 [label="Projects"];
}

\newpage
\subsection{The Form}
The form is a combination of schema and data. The model is composed of dynamic
attributes which have a type. Each attribute is a map of values that have names.
The Form may look like this, represented as JSON.

\begin{lstlisting}
    //Product Form
    {
        id: 3
        name: { 
            type: "string": 
            x: "Red Shoe", 
            y: "Big Red Shoe", 
        },
        description: { 
            type: "string"
            x: "Buy Me", 
            y: "Purchase Me", 
        },
        image: { 
            type: "uri"
            x: "http://a", 
            y: "http://b", 
            z: "http://c",  
        } 
    }
\end{lstlisting}


\newpage
\subsection{The Shadow}
The Product Shadow is what is used when displaying a product to a customer.
At any point in time, the customer has a Context. The Product Shadow is the shape 
you get when you illuminate the Form from a particular vantage point called the Context.  
Here is an example of a Shadow represented as JSON.

\begin{lstlisting}
    //Product Shadow for English Desktop
    {
        context: { modality: "desktop", language: "english" }
        product: 3,
        name: "x",
        description: "x",
        image: "x",
    }
\end{lstlisting}

And here is another, but for mobile...

\begin{lstlisting}
    //Product Shadow for English Mobile
    {
        context: { modality: "mobile", language: "english" }
        product: 3,
        name: "y",
        description: "y",
        image: "z",
    }
\end{lstlisting}

Each attribute in the Shadow corresponds to an attribute in the Form. It selects
the value from the Form via it's name. For example, the 'x' value from the 'name' 
attribute in the Form..

\section{The Context}

Each \emph{Shadow} is a representation of the product \emph{Form} given a \emph{Context}. Each user
will have a Context when they interact with our system. The Context is used
to select which shadows to display for a particular product.

If we assign statistics to each of the Shadow's attributes, as described in 
the companion paper "Everything is Search", this model will provide a solid
foundation for algorithms that could modify shadows or create new ones based on 
data and statistics.



\newpage
\section{Representing SKUs and Variations}
The big question here is how do we map a product to a SKU? And the answer is
the Product Form should have an attribute of type "variations".

\begin{lstlisting}
    {
        variations: {
            type: "variations",
            variation1 : {
                color : {
                    red : "SKU-RED1",
                    green : "SKU-GREEN2",
                }
            }
            variation2 : {
                color : {
                    purple : "SKU-BLUE3",
                    orange : "SKU-ORGAN3",
                }
            }
            ...
        }
\end{lstlisting}

The benefit of this model is we can represent arbitrarily complex variations 
by simply assigning the leaves of the tree with SKU values. If the leaf nodes are
empty, we can have an algorithm that can walk the tree and assign meaningful
SKU numbers.

Also, because of our Shadow model, we can handle cases such as "Only purple and orange
are available in Germany." if the user is in a German context.

\section{Models}

\subsection{Context}

A Context has an identifier, name, and attributes. 
\begin{lstlisting}
    case class Context(
        id: Id,
        name: String,
        attributes: Json)
\end{lstlisting}

Where the attributes can be arbitrarily specific, but should be of a flat key/value nature.

\begin{lstlisting}
    {
        modality: "x",
        language: "y",
        region: "z"
    }
\end{lstlisting}


\subsection{ProductForm}

The ProductForm represents a product as described above. 

\begin{lstlisting}
    case class ProductForm(
        id: Id,
        attributes: Json)
\end{lstlisting}

The attributes should be of the form
\begin{lstlisting}
    {
        attribute1: {
            type: "abc",
            val1 : "x",
            val2 : "y",
            ...
        }
        attribute2: {
            ...
        }
    }
\end{lstlisting}

\subsection{ProductShadow}

The ProductShadow represents a Shadow of a product as described above. 

\begin{lstlisting}
    case class ProductShadow(
        id: Id,
        context: Id,
        productForm: Id,
        attributes: Json)
\end{lstlisting}

Where the attributes correspond to attributes in the product form, selected by
name.

\begin{lstlisting}
    {
        attribute1: "x",
        attribute2: "...",
        ...
    }
\end{lstlisting}


\end{document}
