\documentclass[11pt]{article}

\usepackage{hyperref}
\usepackage{listings}
%for dotex
\usepackage[pdftex]{graphicx}
\usepackage[pdf]{graphviz}
\usepackage[boxed]{algorithm2e} %end for dote
\usepackage{color}

% "define" Scala
\lstdefinelanguage{scala}{
  morekeywords={abstract,case,catch,class,def,%
    do,else,extends,false,final,finally,%
    for,if,implicit,import,match,mixin,%
    new,null,object,override,package,%
    private,protected,requires,return,sealed,%
    super,this,throw,trait,true,try,%
    type,val,var,while,with,yield},
  otherkeywords={=>,<-,<\%,<:,>:,\#,@},
  sensitive=true,
  morecomment=[l]{//},
  morecomment=[n]{/*}{*/},
  morestring=[b]",
  morestring=[b]',
  morestring=[b]"""
}

\lstset{ %
    language=scala,
    identifierstyle=\textbf
}

\title{Inventory Design}
\author{Maxim Noah Khailo}
\begin{document}
\maketitle
\section{Purpose}

An inventory system provides a way to determine if a SKU is in stock. 

\section{Forces}

There are several forces that the architecture of the inventory system has to balance.

\begin{itemize}
        \item Connectability to WMSes.
        \item Connectability to storefronts
        \item Filtering and grouping by SKUs
        \item Filtering and grouping by Catalog information 
        \item Managing an audit trail of inventory changes.
\end{itemize}

The design here will provide a space to balance these forces.

\section{Concepts}

This section provides several concepts which can be combined to provide the functionality 
of the inventory system.

\subsection{Identifiable}

Something you can assign an id to. This is used for equality checking, filtering, and search.

\begin{lstlisting}
    type Id = String
    trait Identifiable {
        def id(): Id
    }
\end{lstlisting}

\subsection{SingularUnit}

Something where all values are of a single unit.

\begin{lstlisting}
    type UnitType = String
    trait SingularUnit {
        def unit(): UnitType
    }
\end{lstlisting}

\subsection{Location, ShippingRestriction}

A trait for filtering based on shipping. This needs to be flushed out more based on
Jeff's shipping restrictions work.

\begin{lstlisting}
    type Location = String
    trait Locatable {
        def location() : Location
    }

    trait HasShippingRestrictions {
        def shippingRestrictions(): Seq[ShippingRestrictions]
    }

\end{lstlisting}

\subsection{OnHand}

Something which can be counted using some units, for example a SKU. 
The setAmount function is used by a WMS for updating. This function
can also be a good place to implement audit trail.

\begin{lstlisting}

    trait OnHand extends Identifiable with SingularUnit{ 
        def onHand(): Int
    }
\end{lstlisting}


\subsection{Handy}

A collection of On Hand things. For example, a Sku

\begin{lstlisting}
    trait Handy {
        def findOnHand(id: Id): Result[OnHand]
    }
\end{lstlisting}

\subsection{Holdable}

An holdable is a trait which modifies adjusts the on hand amount without mutating the original
amount.

\begin{lstlisting}
    trait Holdable extends Identifiable with SingularUnit { 
        def held() : Int
    }
\end{lstlisting}


\subsection{Holdables}

A collection of held things. For example our Inventory system.

\begin{lstlisting}
    trait Holdables { 
        def findHoldable(id: Id): Result[Holdable]
    }
\end{lstlisting}

\subsection{Reservable}

A reservable is something that goes out of availability. 

\begin{lstlisting}
    class Reservable extends Identifiable with SingularUnit{ 
        def reserved() : Result[Int]
    }
\end{lstlisting}

\subsection{Reservables}

A collection of reservable things. 

\begin{lstlisting}
    trait Reservables {
        def findReservable(id: Id): Result[Reservable]
    }
\end{lstlisting}

\subsection{NonSellable}

A non sellable is something that isn't available yet. For example, boxes arrived
in loading dock for a sku.

\begin{lstlisting}
    class NonSellable extends Identifiable with SingularUnit{ 
        def nonSellable() : Result[Int]
    }
\end{lstlisting}

\subsection{NonSellables}

A collection of non sellable things. 

\begin{lstlisting}
    trait NonSellables {
        def findNonSellable(id: Id): Result[NonSellable]
    }
\end{lstlisting}

\subsection{InventoryItem}

A inventory item combines several concepts into one that provides the various values that
make up the inventory calculation.

\begin{lstlisting}
    trait InventoryItem extends 
        OnHand with 
        Reservable with
        Holdable with
        NonSellable{}
\end{lstlisting}

\subsection{InventoryItems}

A collection of inventory items.

\begin{lstlisting}
    trait InventoryItems extends
        OnHand with
        Reservables with
        Holdables with
        NonSellables {

        def findInventoryItem(id: Id): Result[InventoryItem]
    }
\end{lstlisting}

\subsection{Adjustments}

Adjustments are designed to model the ledger which is used to modify inventory items.
The adjustments are applied by a warehouse to inventory items.

\begin{lstlisting}
    object Adjustments {
        sealed trait Adjustment
        case class OnHand(id: Id, amount: Int) extends Adjustment
        case class Held(id: Id, amount: Int) extends Adjustment
        case class Reserved(id: Id, amount: Int) extends Adjustment
        case class NonSellable(id: Id, amount: Int) extends Adjustment
        case class And(left: Adjustment, right: Adjustment)

        type Ledger = Seq[Adjustment]
    }
\end{lstlisting}

\subsection{Warehouse}

A Warehouse is a set of inventory items. It has a location and shipping restrictions.  
It can return inventory items and also apply adjustments to the inventory.

The implementation to this will integrate with a 3rd party WMS and our own inventory 
tracking database.

\begin{lstlisting}
    trait Warehouse extends 
      Identifiable with
      Locatable with
      HasShippingRestrictions with
      InventoryItems {

          def apply(adjustments: Adjustments.Ledger) : Result[Unit]
      
      }
\end{lstlisting}

\subsection{WarehouseManagementSystem}

This is to model the integration with a WMS. At the highest level it supports
an update process which is automatic or manual. It also returns a sequence of warehouses
the WMS manages.

\begin{lstlisting}
    trait WarehouseManagementSystem extends Identifiable { 
        def warehouses() : Seq[Warehouses]
        def update(Warehouse): Future[Result]
    }
\end{lstlisting}

\section{Models}

Given the concepts above we can define some models like SKU, and WMS 

\subsection{InventorySku}

An inventory Sku represents an inventory item. It also computes available for sale of a
sku in a particular warehouse.

\begin{lstlisting}
    case class InventorySku( item: InventoryItem, restrictions: ShippingRestrictions) extends 
        HasShippingRestrictions with InventoryItem {

        def id() = item.id()

        def availableForSale(): Int = 
            math.max(item.onHand() - item.reserved() - item.held(), 0)

        def onHand() = item.onHand()
        def held() = item.held()
        def reserved() = item.reserved()
        def nonSellable() = item.nonSellable()
        def unit() = item.unit()
        def shippingRestrictions() = restrictions
    }
\end{lstlisting}

\subsection{Inventory}

An inventory is a collection of warehouses. It will implement the algorithm that finds
skus in warehouses and computes availble for sale.

\begin{lstlisting}
    class SearchContext(
        l: Option[Location],
        w: Option[Id],
        f: InventoryItem => Boolean) {
            val location = l
            val warehouseId = w
            val filter = f
    }

    case class Inventory(inventoryId: Id, warehouses: Seq[Warehouse]) {
        def id() = inventoryId

        def findInventoryItems( id: Id, c: SearchContext)(implicit ec: ExecutionContext) : Seq[Result[InventoryItem]] = {
            warehouses.map{_.findInventoryItem(id)}
        }

        def findShippingRestrictions(id: Id)(implicit ec: ExecutionContext): Result[ShippingRestrictions] = 
            Result.good(Seq.empty) //TODO: Implement

        def findSkus(id: Id, c: SearchContext)(implicit ec: ExecutionContext) : Seq[Result[InventorySku]] = {
            val restrictions = Seq.empty
            for {
            items ← findInventoryItems(id, c)
            } yield items flatMap {
                case Xor.Right(item) ⇒  Result.good(InventorySku(item, restrictions))
                case Xor.Left(failures) ⇒  Result.failures(failures)
            }
        }
    }
\end{lstlisting}

\subsection{Discussion}

The idea behind the inventory model is to model the half object of the Sku. WMSes do not
keep track of held items from a storefront.

If we trust the on hand amount, we can only hold it on the other side via
a holdable. The other concept is a reservable and this is to model the order reserve transaction
which would send the reserve to a WMS.

We can then have implementations of a warehouse that polls from a WMS creating cached inventory items
in a DB or somewhere else which regularly updates. We can also implement an holdable that
gets its value from a DB stored by the inventory system. Combining the cached and held value
gives you the SKU and amount for sale.

Filtering based on shipping restrictions from and to. We can also do more fancier
things like filtering based on attributes for display purposes.

Notice that you cannot simply find one SKU in the inventory system, but it returns multiple
SKUs as a SKU can be in multiple warehouses.

AFS becomes an accumulation over the SKUs

\section{Architecture}

The inventory system will be composed of an inventory and a set of WMSes. 

\begin{lstlisting}
    class InventorySystem(i: Inventory, 
        ws: Seq[WarehouseManagementSystem]) {

            var inventory = i;
            val wmses = ws

            def afs(sku: Id) = //stuff
            def afs(sku: Id, c: SearchContext) = //stuff
    }

    class InventorySystems(syss: Seq[InventorySystems]) {
        def systems = syss

        def findSystem(id: Id) = //gets system by id
    }

\end{lstlisting}

\subsection{Diagram}
\digraph{Parts} {
    node [shape=record];
    InventorySystem [label="<f0>InventorySystem | {inventory | WMSes}"];
    Inventory [label="<f0>Inventory | {id | warehouses}"];
    Warehouse [label="<f0>Warehouse | {id | find(..)}"];
    WMS [label="<f0>WMS | {warehouses | update(Warehouse)}"];
    ActualWMS [label="Actual WMS"]
    InventorySystem -> Inventory [label="Has One"];
    Inventory -> Warehouse [label="Has Many"];
    InventorySystem -> WMS [label="Has Many"];
    WMS -> Warehouse [label="Has Many"];
    WMS -> ActualWMS [label="Integrates With"];
    Warehouse -> ActualWMS [label="Possibly Integrates With"];
}

\section{TODO}

\begin{itemize}
    \item Todo, flush out shipping restrictions
    \item Figure out how to incorporate catalog and filtering
\end{itemize}

\end{document}
